\documentclass{beamer}

\usepackage[utf8]{inputenc}
\usepackage{default}
\usepackage[T1]{fontenc}%
\usepackage{amssymb}
\usepackage{array}
\usepackage{amsmath}
\usepackage{graphicx}
\usepackage{bbold}
\usepackage[french]{babel}
\usetheme{Berlin}
\useoutertheme{split}
\definecolor{grenat}{RGB}{139,26,26}
\definecolor{beige}{RGB}{255,255,220}

\title{\textbf{Détection du tempo dans un morceau de musique}}
\author{Joshua Peignier \and Estelle Varloot}

\setbeamercolor{structure}{fg = grenat}
\setbeamercolor{background canvas}{bg = beige}

\defbeamertemplate*{footline}{shadow theme}
{%
  \leavevmode%
  \hbox{\begin{beamercolorbox}[wd=.5\paperwidth,ht=2.5ex,dp=1.125ex,leftskip=.3cm plus1fil,rightskip=.3cm]{author in head/foot}%
    \usebeamerfont{author in head/foot}\insertframenumber\,/\,\inserttotalframenumber\hfill\insertshortauthor
  \end{beamercolorbox}%
  \begin{beamercolorbox}[wd=.5\paperwidth,ht=2.5ex,dp=1.125ex,leftskip=.3cm,rightskip=.3cm plus1fil]{title in head/foot}%
    \usebeamerfont{title in head/foot}\insertshorttitle%
  \end{beamercolorbox}}%
  \vskip0pt%
}

\begin{document}
\frame{\titlepage}

\begin{frame}
       \tableofcontents
\end{frame}


\section{Introduction}
\begin{frame}
       \tableofcontents[currentsection]
\end{frame}


\begin{frame}
 \frametitle{Qu'est-ce que le tempo ?}
 \begin{itemize}
  \item<2-> Notion empirique \onslide<3->{$\rightarrow$ Difficilement formalisable}
  \item<4-> Globalement : la fréquence auquel on tape du pied pour accompagner un morceau
  \item<5-> Pour les musiciens : à rapprocher de la pulsation
  \item<6-> Comment extraire le tempo d'un morceau ?
 \end{itemize}
\end{frame}

\begin{frame}
 \frametitle{Problème difficile}
 \begin{itemize}
  \item Intuitivement : le tempo est un objet périodique.
  \item<2-> On peut s'attendre à trouver des pics aux fréquences concernées dans le spectre du signal
  \item<3-> Trop peu d'information exploitable en pratique
  \item<4-> Nécessité de transformer le signal reçu
 \end{itemize}
\end{frame}

\setbeamercolor{background canvas}{bg = white}
\begin{frame}
\begin{center}
 \includegraphics[scale = 0.3]{Algo.png}
\end{center}
 \end{frame}
 
 \setbeamercolor{background canvas}{bg = beige}


 \section{Idées de l'algorithme}
 \begin{frame}
       \tableofcontents[currentsection]
\end{frame}


 \begin{frame}
  \begin{itemize}
  \item Information intéressante : \onslide<2->{Intuitivement, l'enveloppe du signal}
  \item<3-> Selon le morceau : plusieurs enveloppes possibles
  \item<4-> Nécessité de séparer le spectre en bandes de fréquences
  \item<5-> Utilisation d'un banc de filtres
  \item<6-> Extraction d'enveloppe par bandes
  \end{itemize}
 \end{frame}

 \begin{frame}
  \begin{itemize}
   \item Le tempo : deux informations
   \begin{itemize}
   \item<2-> La fréquence de la pulsation
   \item<3-> La phase de la pulsation
   \end{itemize}
   \item<4-> Pour la détection : nécessiter de trouver la phase
   \item<5-> Algorithme utilisé : développé empiriquement
  \end{itemize}

 \end{frame}

 \begin{frame}
  \begin{itemize}
   \item Division du signal en 6 bandes spectrales à l'aide d'un banc de filtres
   \item<2-> Pour chaque sortie : calcul de l'enveloppe du signal
   \item<3-> Calcul de la dérivée temporelle de chaque enveloppe
   \item<4-> Transmission des données à des bancs de filtres résonateurs
   \item<5-> Extraction de signaux en phase avec le signal d'origine
  \end{itemize}
 \end{frame}

 \section{Séparation en bandes et extraction d'enveloppe}
 
  \begin{frame}
       \tableofcontents[currentsection]
\end{frame}

 \setbeamercolor{background canvas}{bg = white}
 \begin{frame}
    \frametitle{L'implémentation de Scheirer}
    \begin{center}
   \includegraphics[scale = 0.3]{filterbank.png}
   \end{center}
 \end{frame}
 \setbeamercolor{background canvas}{bg = beige}
 
  \begin{frame}
    %\begin{center}
    \frametitle{Notre implémentation}
   \includegraphics[scale = 0.26]{1stband.png}
   \includegraphics[scale = 0.257]{2ndband.png}
     \begin{itemize}
   \item<2-> Réalisée à l'aide de filtres elliptiques
  \end{itemize}
   %\end{center}
 \end{frame}
 
 \section{Détection d'attaque}
   \begin{frame}
       \tableofcontents[currentsection]
\end{frame}

 \begin{frame}
  \begin{itemize}
   \item Pour chacune des sorties : convolution avec une fenêtre de Hann
   \onslide<2->{
    \begin{eqnarray*}
h(t) & = &  \mathbb{1}_{[0,T]}\frac{1}{2}(1-cos(2\pi\frac{t}{T}))
 \end{eqnarray*}
   }
  \end{itemize}
  \begin{itemize}
  \item<3-> Principe : Masquer les modulations rapides
  \end{itemize}
 \onslide<4->{$\rightarrow$ Extraction d'une enveloppe}
 \end{frame}

 \begin{frame}
  \begin{itemize}
   \item Dérivation et décimation
   \item<2-> Encore à implémenter
  \end{itemize}

 \end{frame}

 \section{Filtres résonateurs}
   \begin{frame}
       \tableofcontents[currentsection]
\end{frame}

\begin{frame}
\begin{itemize}
 \item Un banc de filtres
 \item<2-> Balayage de fréquence pour trouver les différents tempos possibles
 \item<3-> Sommation des différentes sorties
 \item<4-> Recherche de l'énergie maximale pour trouver le tempo
\end{itemize}

 
\end{frame}

 
 \section{Conclusion}
  \begin{frame}
       \tableofcontents[currentsection]
\end{frame}

 \begin{frame}
  \begin{itemize}
   \item Implémenté :
   \begin{itemize}
    \item<2-> Le banc de filtres séparateurs
   \end{itemize}
   \item<3-> A faire : \onslide<4->{Tout le reste}
    \begin{itemize}
      \item<5-> La convolution avec la fenêtre de Hann 
    \end{itemize}
  \end{itemize}

 \end{frame}


\end{document}
